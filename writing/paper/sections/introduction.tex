The study of permutation patterns began as a result of Knuth's statements on
stack sorting in \emph{The Art of Computer Programming}~\cite[p.~243,
Ex.~5,6]{Knuth:1997:ACP:260999}. The original concept---a subsequence of
symbols having a particular relative order, now known as classical
patterns---has been expanded to a variety of definitions.
\textcite{babstein2000} considered \emph{vincular} patterns (also known as
\emph{generalised} or \emph{dashed} patterns) where two adjacent entries in the
pattern can be required to be adjacent in the permutation. \textcite{MR2652101}
look at classes of patterns where entries can also be required to be consecutive in
value, these are called \emph{bivincular} patterns. \emph{Bruhat-restricted}
patterns were studied by \textcite{MR2264071} to establish necessary conditions
for a Schubert variety to be Gorenstein. These definitions are subsumed under
the definition of \emph{mesh patterns}, introduced by
\textcite{journals/combinatorics/BrandenC11} to capture explicit expansions for
certain permutation statistics.

When considering permutation patterns some of the main questions posed relate to
how and when a pattern is avoided by, or contained in, an arbitrary set of
permutations. Two patterns \(\pi\) and \(\sigma\) are \emph{Wilf-equivalent} if
the number of permutations that avoid \(\pi\) of length \(n\) is equal to the
number of permutations that avoid \(\sigma\) of length \(n\). A stronger
equivalence condition is that of \emph{coincidence}, where the set of
permutations avoiding \(\pi\) is exactly equal to the set of permutations
avoiding \(\sigma\). Avoiding pairs of patterns of the same length with certain
properties has been studied, \textcite{MR2178749} considered avoiding
a pair of vincular patterns of length 3. \textcite{2015arXiv151203226B} study
avoiding a vincular and a covincular pattern simultaneously in order to achieve
several counting results. However, little work has been done on
avoiding a mesh pattern and a classical pattern simultaneously.

In this work we aim to establish some ground in this field by computing
coincidences and Wilf-classes and calculating some of the enumerations of
avoiders of a mesh pattern of length 2 and a classical pattern of length 3. We
begin by establishing coincidences between mesh patterns of length 2 while
avoiding a dominating pattern by computational methods, which are then
used to establish three ``Dominating Pattern Rules'' as well as some special cases
that can be used to calculate coincidences.

 We then use these coincidence classes
to calculate Wilf-equivalence classes showing some of the methods used.
