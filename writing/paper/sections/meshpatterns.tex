A \emph{permutation} is a bijection from the set \(\nrange{n} = \setrange{1}{n}\) to itself. The
set of all such bijections on this set if denoted \(\mathfrak{S}_n\) and has size \(n!\). We can
denote an individual permutation \(\pi\in\mathfrak{S}_n\) in \emph{one-line notation} by writing the
entries of the permutation in order, therefore \(\pi = \perm{\pi(1),\pi(2),\dotso,\pi(n)}\). The
set \(\mathfrak{S}_n\) has exactly one element, the empty permutation \(\varepsilon\).

\begin{definition}{(Order isomorphism.)}
    Two substrings \(\alpha_1\alpha_2\dotsm\alpha_n\) and
    \(\beta_1\beta_2\dotsm\beta_n\) are said to be \emph{order isomorphic}
    if they share the same relative order, \ie, \(\alpha_r<\alpha_s\) if and
    only if \(\beta_r<\beta_s\).
\end{definition}

The definition of order isomorphism allows us to give the meaning of containment for classical
permutation patterns.
\begin{definition}
    A permutation \(\pi\in\mathfrak{S}_n\) contains the pattern \(\sigma\in\mathfrak{S}_k\) (\mbox{denoted} 
    \(\sigma \preceq \pi\)) if there is some subsequence of indices of \(\pi, i_1i_2\dotsm{}i_k\)
    such that the sequence \(\pi(i_1)\pi(i_2)\dotsm\pi(i_k)\) is order isomorphic to
    \(\sigma(1)\sigma(2)\dotsm\sigma(k)\). If \(\pi\) does not contain \(\sigma\), we say that 
    \(\pi\) \emph{avoids} \(\sigma\).
\end{definition}
\begin{example}
\label{ex:contexmpl}
The permutation \(\pi = \perm{2,4,1,5,3}\) contains the pattern 
\(\sigma = \perm{2,3,1}\), since the second, fourth and fifth elements (\(453\)) are order 
isomorphic to \(\perm{231}\), it also contains the occurrence \(241\). The permutation 
\(\perm{2,4,1,5,3}\) avoids the pattern \(\perm{3,2,1}\).
\end{example}

We denote the set of permutations of length \(n\) avoiding a pattern \(\sigma\) as \(\Av_n(\sigma)\)
and \(\av{\sigma}=\bigcup_{i=0}^{\infty}\Av_i(\sigma)\).

We can display a permutation graphically in a \emph{plot}, in such a plot we display the points
\(G(\pi) = \{(i,\pi(i))\mid i\in[1,n]\}\) in a Cartesian coordinate system. The plots of the permutations \(\pi = 
\perm{2,4,1,5,3}\) and \(\sigma = \perm{2,3,1}\) can be seen in \FigureRef{fig:plots}.
\FigureRef{fig:containment} shows the containment of \(\sigma\) in \(\pi\) as in \ExampleRef{ex:contexmpl}.

\begin{figure}[htb]
    \begin{center}
    \raisebox{6ex}{\(G(\pi)=\)}
    \begin{tikzpicture}[scale=\picscale]
        \modpattern{}{2,4,1,5,3}{}
    \end{tikzpicture}
    \raisebox{6ex}{\(\quad{}G(\sigma)=\)}
    \raisebox{0.35cm}{
    \begin{tikzpicture}[scale=\picscale]
        \modpattern{}{2,3,1}{}
    \end{tikzpicture}}

        \caption{The plots of the permutations \(\pi\) and \(\sigma\).}
        \label{fig:plots}
    \end{center}
\end{figure}

\begin{figure}[htb]
    \begin{center}
    \(\pattern{2,4,5}{2,4,1,5,3}{}\)
        \caption{The occurrence of 231 in 24153 corresponding to 453.}
        \label{fig:containment}
    \end{center}
\end{figure}

\begin{definition}
A \emph{mesh pattern} is a pair
\begin{equation*}
    p = (\tau,R)\text{ with } \tau \in \mathfrak{S}_k \text{ and } R \subseteq [0,k]\times [0,k].
\end{equation*}
\end{definition}
We define containment (denoted \(p\preceq \pi\)), and avoidance, of the pattern \(p\) in the permutation
\(\tau\) on mesh patterns analogously to classical containment, and avoidance,
of \(\pi\) in \(\tau\) with the additional restrictions on the relative
position of the occurrence of \(\pi\) in \(\tau\). These restrictions say that
no elements of \(\tau\) are allowed in the regions of the plot corresponding
to shaded boxes in the mesh.

Formally defined by \textcite{journals/combinatorics/BrandenC11},
an \emph{occurrence} of \(p\) in \(\tau\) is a subset \(\omega\) of the plot of
\(\tau, G(\tau) = \{(i,\tau(i)\mid i\in[1,n]\}\) such that there are order-preserving
injections \(\alpha,\beta:[1,k]\mapsto[1,n]\) satisfying the following two conditions.

\noindent Firstly, \(\omega\) is an occurrence of \(\pi\) in the classical sense
  \begin{enumerate}[i.]
    \item \(\omega = \{(\alpha(i),\beta(j)):(i,j)\in G(\pi)\}\)
  \end{enumerate}\saveenum
  Define \(R_{ij} = [\alpha(i)+1,\alpha(i+1)-1]\times[\beta(j)+1,\beta(j+1)-1]\)
   for \(i,j\in[0,k]\) where \(\alpha(0)=\beta(0)=0\) and \(\alpha(k+1)=\beta(k+1)=n+1\). Then
   the second condition is
  \begin{enumerate}[i.]\resetenum
  \item if \((i,j) \in R \text{ then } R_{ij} \cap G(\tau) = \emptyset\)
\end{enumerate}
We call \(R_{ij}\) the region corresponding to \((i,j)\).
We define containment of a mesh pattern \(p\) in another mesh \(\kappa\) as above,
with the additional condition that if \((i,j)\in R\text{ then } R_{ij}\) is
contained in the mesh set of \(\kappa\), in this case we call \(p\) a \emph{subpattern}
of \(\kappa\).

\begin{example}
    The pattern \(p=\mperm{2,1,3}{\{(0,1),(0,2),(1,0),(1,1),(2,1),(2,2)\}}=
    \textpattern{}{ 2, 1, 3 }{ 0/1, 0/2, 1/0, 1/1, 2/1, 2/2 }\) is contained in
    \(\pi = \perm{3,4,2,1,5}\) but is not contained in \(\sigma = \perm{4,2,3,1,5}\).
\end{example}

Let us consider the plot for the permutation \(\pi\). The subsequence \(325\) is
an occurrence of \(213\) in the classical sense and the remaining points of \(\pi\)
are not contained in the regions corresponding to the shaded boxes in \(p\).
\begin{equation*}
    \pattern{1,3,5}{3,4,2,1,5}{0/2,0/3,0/4,
                               1/0,1/1,1/2,
                               2/0,2/1,2/2,
                               3/2,3/3,3/4,
                               4/2,4/3,4/4}
\end{equation*}
The subsequence \(325\) is therefore an occurrence of the pattern \(p\) in \(\pi\) and
\(\pi\) contains \(p\).

Now we consider the plot for the permutation \(\sigma\). This permutation
avoids \(p\) since for every occurrence of the classical pattern \(\perm{2,1,3}\)
there is at least one point in one of the shaded boxes. For example, consider the
subsequence \(315\) in \(\sigma\), this is an occurrence of \(\perm{2,1,3}\) but
not the mesh pattern since the points with values \(4\) and \(2\) are in the
regions corresponding to the boxes \((0,1)\) and \((0,2)\), which are shaded in
\(p\). This is shown in the plot below.

\begin{equation*}
    \pattern{3,4,5}{4,2,3,1,5}{0/1,0/2,0/3,0/4,
                               1/1,1/2,1/3,1/4,
                               2/1,2/2,2/3,2/4,
                               3/0,3/1,3/2,
                               4/1,4/2,4/3,4/4}
\end{equation*}
This is true for all occurrences of \(\perm{2,1,3}\) in \(\sigma\) and
therefore \(\sigma\) avoids \(p\).

We denote the avoidance sets for mesh patterns in the same way as for
classical patterns. Given a mesh pattern \(p=(\sigma,R)\) we say that \(\sigma\) is the \emph{underlying
classical pattern} of \(p\).