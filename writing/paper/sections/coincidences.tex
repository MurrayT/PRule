Coincidences among small mesh patterns have been considered by
\textcite{DBLP:journals/corr/ClaessonTU14}, in which the authors use the Simultaneous Shading Lemma,
a closure result and one worked out special case to fully classify coincidences among mesh patterns
of length 2.

Two patterns \(\lambda\) and \(\gamma\) are considered \emph{coincident} if the set of permutations
that avoid \(\lambda\) is the same as the set of permutations that avoid \(\gamma\),
\ie \(\av{\lambda} = \av{\gamma}\). Equivalently we can say that they have the same set of
\emph{containers},\ie \(\cont{\lambda} = \cont{\gamma}\).

We will consider the avoidance sets \(\av{\pi,p}\) where \(\pi\) is a classical pattern of length 3
and \(p\) is a mesh pattern of length 2 in order to establish sufficient conditions for two such
sets to be coincident. We will fix \(\pi\) in order to define these coincidences and say that
\(\pi\)is the \emph{dominating pattern}.

In order to describe the rules it is useful to have a notion for inserting points, ascents, and
descents into a mesh pattern.
\begin{definition}
\label{def:ap}
Let \(p=(\tau,R)\) be a mesh pattern of length \(n\) such that \(\boks{i,j}\notin R\). We define
a mesh pattern \(p^\boks{i,j} = (\tau^\prime,R^\prime)\) of length \(n+1\) as the pattern where a
point is \emph{inserted} into the box \(\boks{i,j}\) in \(G(p)\). Formally the new underlying
classical pattern is defined by
\begin{equation*}
\tau^\prime(k) = \begin{cases}
    j+1 & \text{if } k = i+1\\
    \tau(k) & \text{if } \tau(k)\le j \text{ and }k\le i\\
    \tau(k)+1 & \text{if } \tau(k)> j \text{ and }k\le i\\
    \tau(k-1) & \text{if } \tau(k)\le j \text{ and }k> i+1\\
    \tau(k-1)+1 & \text{if } \tau(k)> j \text{ and }k> i+1\\
\end{cases}
\end{equation*}
While the mesh becomes
\begin{equation*}
\begin{aligned}
R^\prime=&\{\boks{k,\ell}\mid k\le i, \ell\le j, \boks{k,\ell}\in R\} \cup\\
&\{\boks{k,\ell}\mid k\le i, \ell > j, \boks{k,\ell-1}\in R\}\cup\\
&\{\boks{k,\ell}\mid k > i, \ell \le j, \boks{k-1,\ell}\in R\}\cup\\
&\{\boks{k,\ell}\mid k > i, \ell > j, \boks{k-1,\ell-1}\in R\}\\
\end{aligned}
\end{equation*}
\end{definition}
In addition, we give the following definitions:
\begin{definition}
Let \(p=(\tau,R)\) be a mesh pattern of length \(n\) such that \(\boks{i,j}\notin R\) and
\(p^\boks{i,j} = (\tau^\prime,R^\prime)\) to be as defined in \DefinitionRef{def:ap}. We define
the following five modifications of the mesh patterns of the same length as \(p^\boks{i,j}\).
\begin{align*}
p^\boksup{i,j} &= (\tau^\prime, R^\prime \cup \{\boks{i,j+1},\boks{i+1,j+1}\})\\
p^\boksright{i,j} &= (\tau^\prime, R^\prime \cup \{\boks{i+1,j},\boks{i+1,j+1}\})\\
p^\boksdown{i,j} &= (\tau^\prime, R^\prime \cup \{\boks{i,j},\boks{i+1,j}\})\\
p^\boksleft{i,j} &= (\tau^\prime, R^\prime \cup \{\boks{i,j},\boks{i,j+1}\})\\
\end{align*}
\end{definition}
Informally, these are considering the topmost, rightmost, leftmost, or bottommost point in
\(\boks{i,j}\). We allow composition of these modifications, and collect the resulting mesh patterns
in a set
\begin{equation*}
p^\boksall{i,j}=\left\{
p^\boks{i,j},
p^\boksright{i,j},
% p^\boksrightup{i,j},
% p^\boksrightdown{i,j},
p^\boksleft{i,j},
% p^\boksleftup{i,j},
% p^\boksleftdown{i,j},
p^\boksup{i,j},
p^\boksdown{i,j}\right\}
\end{equation*}
See \FigureRef{fig:addp} for an example of adding a point into a mesh pattern.

\begin{figure}
\begin{equation*}
%\adda{\pattern{}{2,3,1}{0/0,1/0,1/3,
%                        3/0,3/1,3/3}}{(1,1)}
p = \pattern{}{1,2}{0/1,2/2} \quad
p^\boksrightup{2,1}
= \pattern{}{1,3,2}{0/1,0/2,2/2,2/3,3/1,3/2,3/3}
\end{equation*}
\caption{The result of inserting a point into \(p=(12,\{(0,1),(2,2)\})\)}
\label{fig:addp}
\end{figure}

\begin{definition}
Let \(p=(\tau,R)\) be a mesh pattern of length \(n\) such that \(\boks{i,j}\notin R\). We define
a mesh pattern \(p^{\boks{i,j}_a} = (\tau^\prime,R^\prime)\) (\(p^{\boks{i,j}_d}\)) of length \(n+2\)
as the pattern where an ascent (descent) is \emph{inserted} into the box \(\boks{i,j}\) in \(G(p)\).
Formally the new underlying classical pattern is defined by
\begin{equation*}
\tau^\prime(k) = \begin{cases}
    j+t & \text{if } k = i+t,t\in\{1,2\}\\
    \tau(k) & \text{if } \tau(k)\le j \text{ and }k\le i\\
    \tau(k)+2 & \text{if } \tau(k)> j \text{ and }k\le i\\
    \tau(k-2) & \text{if } \tau(k)\le j \text{ and }k> i+2\\
    \tau(k-2)+2 & \text{if } \tau(k)> j \text{ and }k> i+2\\
\end{cases}
\end{equation*}
The ordering of the top branch determines whether an ascent(or descent) is added.
The mesh becomes
\begin{equation*}
\begin{aligned}
R^\prime=&\{\boks{k,\ell}\mid k\le i, \ell\le j, \boks{k,\ell}\in R\} \cup\\
&\{\boks{k,\ell}\mid k\le i, \ell > j, \boks{k,\ell-2}\in R\}\cup\\
&\{\boks{k,\ell}\mid k > i, \ell \le j, \boks{k-2,\ell}\in R\}\cup\\
&\{\boks{k,\ell}\mid k > i, \ell > j, \boks{k-2,\ell-2}\in R\}\cup\\
&\{\boks{i,j+1},\boks{i+1,j},\boks{i+1,j+1},\boks{i+1,j+2},\boks{i+2,j+1}\}
\end{aligned}
\end{equation*}
\end{definition}
An example of adding an ascent to a mesh pattern can be seen in \FigureRef{fig:adda}.
\begin{figure}
\begin{equation*}
p = \pattern{}{2,3,1}{0/0,1/0,1/3,
                      3/0,3/1,3/3} \quad
p^{\boks{1,1}_a}= \pattern{}{4,2,3,5,1}{0/0,
                        1/0,1/2,1/5,
                        2/0,2/1,2/2,2/3,2/5,
                        3/0,3/2,3/5,
                        5/0,5/1,5/2,5/3,5/5}
\end{equation*}
\caption{The result of inserting an ascent into \(p=(231,\{(0,0),(1,0),(1,3),(3,0),(3,1),(3,3)\})\)}
\label{fig:adda}
\end{figure}

We now attempt to fully classify coincidences in families characterised by avoidance
of a classical pattern of length \(3\) and a mesh pattern of length \(2\), that is
finding and explaining all coincidences where \(\av{\{p,m\}} = \av{\{p,m^\prime\}}\).

It can be easily seen that in order to classify coincidences one need only
consider coincidences within the family of mesh patterns with the same underlying
classical pattern, this is due to the fact that \(\perm{2,1}\in\av{\mperm{1,2}{R}}\)
and \(\perm{1,2}\in\av{\mperm{2,1}{R}}\) for all mesh-sets \(R\).

We know that there are a total of \(512\) mesh-sets for each underlying classical
pattern. By use of the previous results of \textcite{DBLP:journals/corr/ClaessonTU14}\footnote{
The authors use the Simultaneous Shading Lemma, a closure result and one worked out special case.
} the number of coincidence classes can be reduced to \(220\).

\subsection{Coincidence classes of Av(\{321, (21, \textit{R})\})}
Through experimentation, considering avoidance of permutations of up to length \(11\), we discover
that there are at least \(29\) coincidence classes of mesh patterns with underlying classical
pattern \(21\).

\begin{proposition}[First Dominating Pattern Rule]
    \label{prop:dom1}
    Given two mesh patterns \(m_1 =(\sigma, R_1)\) and \(m_2 = (\sigma, R_2)\),
    and a dominating classical pattern \(\pi = (\pi,\emptyset)\) such that
    \(\setsize{\pi} \le \setsize{\sigma} + 1\), the sets \(\av{\{\pi,m_1\}}\) and
    \(\av{\{\pi,m_2\}}\) are coincident if

    \begin{enumerate}
        \item \(R_1 \triangle R_2 = \{(a,b)\}\)
        \item \(\pi \preceq \sigma^\boks{a,b}\)\label{prop:dom1:cont}
    \end{enumerate}
\end{proposition}
In order to prove this proposition we must first make the following note.

\begin{note}
    \label{not:downcmesh}
    Let \(R^\prime \subseteq R\). Then any occurrence of \((\tau, R)\) in a permutation
    is an occurrence of \((\tau,R^\prime)\).
\end{note}

\begin{proof}[Proof of \PropositionRef{prop:dom1}]
    We need to prove that \(\av{\{\pi,m_1\}} = \av{\{\pi,m_2\}}\).

    \noindent Assume without meaningful loss of generality that \(R_2 = R_1 \cup \{(a,b)\}\).
    Since \(R_1\) is a subset of \(R_2\), \NoteRef{not:downcmesh} states that
    \(\av{\{\pi,m_1\}} \subseteq \av{\{\pi,m_2\}}\)

    Now we consider a permutation \(\omega^\prime\in \av{\pi}\),
    containing an occurrence of \(m_1\). Consider placing a point in the
    region corresponding to the box \((a,b)\), regardless of where in this
    region we place the point by condition~\ref{prop:dom1:cont} of the
    Proposition we create an occurrence of \(\pi\), therefore there can be no
    points in this region, which could have been represented in the mesh set
    \(R_1\) by adding the box \((a,b)\). Hence every occurrence of
    \(m_1\) is in fact an occurrence of \(m_2\), and we have that
    \(\av{\{\pi,m_2\}} \subseteq \av{\{\pi,m_1\}}\).

    Taking both directions of the containment we can therefore draw the
    conclusion that \(\av{\{\pi,m_1\}} = \av{\{\pi,m_2\}}\).
\end{proof}
