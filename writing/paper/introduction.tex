The study of permutation patterns began as a result of Knuth's statements on stack sorting in
\emph{The Art of Computer Programming}\cite[p.~243, Ex.~5,6]{Knuth:1997:ACP:260999}. This original
concept---a subsequence of symbols having a particular relative order, now known as classical
patterns---has been expanded to a variety of definitions. Mesh patterns \textcite{babstein2000} 
considered \emph{vincular} patterns---also known as \emph{generalised} or \emph{dashed} 
patterns---where two adjacent entries in the pattern must also be adjacent in the permutation. 
\textcite{MR2652101} look at classes of pattern where both columns and rows can be shaded, these are
called \emph{bivincular} patterns. \emph{Bruhat-restricted} patterns were studied by 
\textcite{MR2264071} in order to establish necessary conditions for a Schubert variety to be
Gorenstein. All of these definitions are subsumed under the definition of \emph{mesh patterns},
introduced by \textcite{journals/combinatorics/BrandenC11} to capture explicit expansions for 
certain permutation statistics.

When considering permutation patterns some of the main questions posed relate to how and when a
pattern is avoided by, or contained in, a arbitrary set of permutations. Two patterns \(\pi\) and
\(\sigma\) are \emph{Wilf-equivalent} if the number of permutations that avoid \(\pi\) of length 
\(n\) is equal to the number of permutations that avoid \(\sigma\) of length \(n\). A stronger 
equivalence condition is that of \emph{coincidence}, where the set of permutations avoiding \(\pi\)
is exactly equal to the set of permutations avoiding \(\sigma\). Avoiding pairs of patterns of the
same length with certain properties has been studied previously, \textcite{MR2178749} considered 
avoiding a pair of vincular patterns of length 3. \textcite{2015arXiv151203226B} study avoiding a 
vincular and a covincular pattern simultaneously in order to achieve some interesting counting 
results. However, very little work has been done on avoiding a mesh pattern and a classical pattern 
simultaneously. 

In this work we aim to establish some ground in this field by computing coincidences
and Wilf-classes and calculating some of the enumerations of avoiders of a mesh pattern of length
2 and a classical pattern of length 3. We begin by establishing coincidences between mesh patterns
of length 2 while avoiding a classical pattern of length 3, this is used to establish sufficient
conditions for coincidence. We then establish Wilf-equivalence classes of these coincidence classes
who avoid the classical pattern 231.