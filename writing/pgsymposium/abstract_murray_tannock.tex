\section{Equivalence classes of Mesh patterns with a dominating
pattern}\label{equivalence-classes-of-mesh-patterns-with-a-dominating-pattern}

\subsection{Murray Tannock}\label{murray-tannock}

A permutation is an arrangement of \(n\) objects. We can define sets of
permutations by the avoidance of subsequences with certain properties.
One of these is \emph{classical pattern avoidance} where the values of
the entries in the subsequence must not have a particular ordering.
Classical patterns capture interesting properties of permutations such
as stack sortability, and as such have links to different combinatorial
objects. Mesh patterns are an extension of classical patterns that allow
more restrictions on the occurrence of the patterns. We say that two
sets of patterns are \emph{coincident} if they are avoided by the same
set of permutations. We establish sufficient conditions for coincidence
of mesh patterns, whilst also avoiding a longer classical pattern. These
conditions, along with two special cases, completely classify
coincidence between families containing a mesh pattern of length \(2\)
and a classical pattern of length \(3\).

(Supervised by Michael Albert)
