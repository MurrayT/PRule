\section{Wilf-equivalences}
\label{sec:Wilf-equivalences}
\subsection{Wilf-equivalence}
\label{sub:Wilf-equivalence}
\begin{frame}{Wilf-equivalence}
%   \begin{definition}
%   Two patterns are said to be \emph{Wilf-equivalent} if the set of avoiders
%   of the patterns is the same size at every length.
% \end{definition}
 All coincident pattern classes are Wilf-equivalent.

 We Wilf-classify classes of the form \(\av{231,m}\) where \(m\)
 is a mesh pattern.

 % The permutations in \(\av{\perm{2,3,1}}\) of length \(n\) are counted by the
 % Catalan numbers.
 % \begin{equation}
 %   C_0 = 1, \text{ and } C_{n+1} =\sum_{i=0}^{n}C_iC_{n-i}
 % \end{equation}
 % This gives the sequence
 % \begin{equation*}
 %   1,1,2,5,14,42,132,1430,4862,\dotsc
 % \end{equation*}
\end{frame}
\subsection{Set Partitions}
\label{sub:Set Partitions}
\begin{frame}{Set Partitions}
  \begin{block}{Note}
    The avoiders of the pattern \(q = \mperm{2,3,1}{\{(1,0),(1,1),(1,2),(1,3)\}}\),
    \textpattern{}{2,3,1}{1/0,1/1,1/2,1/3},
    in \(\mathfrak{S}_n\) are in one-to-one correspondence with partitions of
    \(\nrange{n}\). (Claesson \cite{A})
\end{block}
\begin{example}
    Given the permutation \(\pi = \perm{5,4,2,1,3,9,6,8,7}\) this corresponds to the
    partition \(\{\{5,4,2,1\},\{3\},\{9,6\},\{8,7\}\}\).
\end{example}
All permutations in \(\av{\perm{2,3,1}}\) are also in \(\av{q}\).
\end{frame}
\begin{frame}
  Set Partitions can be used to explain certain Wilf-equivalences.
  \begin{example}
    The following patterns are experimentally Wilf-equivalent up to length 10 in
\(\av{\perm{2,3,1}}\)
  \begin{equation*}
      m_1 = \pattern{}{1,2}{0/0,0/2,
                      1/0,1/1,
                      2/0,2/1,2/2}
      \text{ and }
      m_2 = \pattern{}{1,2}{0/0,
                      1/0,1/1,1/2,
                      2/0,2/1,2/2}
  \end{equation*}
  Consider containers of the patterns in \(\av{\perm{2,3,1}}\).

  \only<2>{There can only ever be a single occurrence of \(m_1\). The shading
  shows us that this is a partition of \(k\) elements into two non-overlapping
  parts by the first element and the last element.}
  \only<3>{There can only ever be a single occurrence of \(m_2\). The shading
  shows us that this is a partition of \(k\) elements into two non-overlapping
  parts, where the split in parts is determined by the value of the last element
  in the permutation.}
  \only<4->{Therefore for each of these patterns a container of length \(k\)
  consists of a decreasing sequence of length \(k-2\) split into two parts.
  There are \(k-1\) ways to perform such a split. So the number of avoiders of
  length \(k\) is
  \begin{equation*}
    K_k = C_k - (k-1), K_0 = 1
  \end{equation*}}
  \end{example}

\end{frame}
\subsection{Generating Functions}
\label{sub:Generating Functions}
% \begin{frame}{Generating {\huge FUN}ctions}
%   \begin{block}{}
%     \blockquote[Herbert Wilf, \emph{generatingfunctionology}~\cite{Wilf}]{%
%     A generating function is a clothesline on which we hang up a sequence of numbers for display.}
%     % \flushright -Herbert Wilf, \emph{generatingfunctionology}
%   \end{block}
%   \begin{definition}
%     The \emph{ordinary generating function (OGF)} of a sequence \((A_n)\) is the
%     formal power series
%     \begin{equation}
%       A(x) = \sum_{n=0}^{\infty} A_nx^n
%     \end{equation}
%     The right hand side is the Taylor expansion of the left hand side at \(0\).
%   \end{definition}
% \end{frame}
% \begin{frame}
%   \begin{example}
%     The generating function for the constant sequence \(\{1,1,1,1,\dotsc\}\) is
%     \begin{equation*}
%       f(x) = \frac{1}{1-x}
%     \end{equation*}
%   \end{example}
% \end{frame}
\begin{frame}{Generating Functions by Structure}
  \begin{example}
  The generating function for the avoiders of the permutation \(\perm{2,3,1}\)
  satisfies
  \begin{equation}
    C(x) = 1 +xC^2(x)
  \end{equation}
  This can be seen structurally (\(\mathcal{C}\) is the set of \(\perm{2,3,1}\) avoiders)
  \begin{figure}[!ht]
    \centering
    \(\mathcal{C} = \varepsilon \mathrel{\sqcup}\) \decompleft{0/1,1/0}{{\mathcal{C}}}{}{}{{\mathcal{C}}}
\end{figure}
% \only<2->{
% Solving for \(C(x)\) and taking the Taylor expansion at \(0\) gives
% \begin{equation*}
%   1+x+2x^2+5x^3+14x^4+42x^5+132x^6+1430x^7+4862x^8+\dotsb
% \end{equation*}
% }
%
\end{example}
\end{frame}
\begin{frame}
  Now we consider one of the Wilf-equivalences between mesh patterns in \(\av{\perm{2,3,1}}\).
  \begin{equation*}
    m_1 = \pattern{}{1,2}{1/0,1/1,1/2,2/0,2/1} \text{ and }
    m_2 = \pattern{}{2,1}{0/1,1/0,1/1,1/2,2/1}
  \end{equation*}
\end{frame}

\begin{frame}
  First we consider the avoiders \(\mathcal{M}_1\) of \(m_1 = \pattern{}{1,2}{1/0,1/1,1/2,2/0,2/1} \) in \(\av{\perm{2,3,1}}\).
  \begin{equation*}
    \mathcal{M}_1 = \varepsilon \sqcup
    \decompleft{0/1,1/0}{\mathcal{M}_1}{}{}{\mathcal{M}^\prime_1}
\end{equation*}
Where \(\mathcal{M}^\prime_1\) is a permutation avoiding \(\perm{2,3,1}, m_1\)
and \(\textpattern{}{1}{0/0,0/1,1/0}\). \pause Now considering a permutation in \(\mathcal{M}^\prime_1\)
\begin{equation*}
    \mathcal{M}^\prime_1 = \varepsilon \sqcup
    \decompleft{0/1,1/0}{\scriptstyle \mathcal{M}_1\setminus\varepsilon}{}{}{\mathcal{M}^\prime_1}
\end{equation*}
\end{frame}

\begin{frame}
  Next we consider the avoiders \(\mathcal{M}_2\) of \(m_2 = \pattern{}{2,1}{0/1,1/0,1/1,1/2,2/1}\) in \(\av{\perm{2,3,1}}\).
  \begin{equation*}
    \mathcal{M}_2 = \varepsilon \sqcup
    \decompleft{0/1,1/0}{\mathcal{M}^\prime_2}{}{}{\mathcal{M}_2}
\end{equation*}
Where \(\mathcal{M}^\prime_2\) is a permutation avoiding \(\perm{2,3,1}, m_2\)
and \(\textpattern{}{1}{0/0,0/1,1/1}\). \pause Now considering a permutation in \(\mathcal{M}^\prime_2\)
\begin{equation*}
    \mathcal{M}^\prime_2 = \varepsilon \sqcup
    \decompleft{0/1,1/0}{\mathcal{M}^\prime_2}{}{}{\scriptstyle \mathcal{M}_2\setminus\varepsilon}
\end{equation*}
\end{frame}
\begin{frame}
  Both of these sets therefore have generating function \(M(x)\) satisfying
  \begin{align}
    M(x) &= 1 + xM(x)M^\prime(x) \label{eqn:Pgen}\\
    M^\prime(x) &= 1 + x(M(x)-1)M^\prime(x)\label{eqn:PprimeGen}
 \end{align}\pause
 Solving \ref{eqn:PprimeGen} for \(M^\prime(x)\) and substituting into
\ref{eqn:Pgen} gives us that the generating function for
\(M(x)\) satisfies
\begin{equation}
    M(x) = xM^2(x) - x(M(x) - 1) + 1
\end{equation}\pause
Evaluating coefficients of \(M(x)\) then gives the sequence
\begin{equation*}
    1, 1, 1, 2, 4, 9, 21, 51, 127, 323, 835,\dotsc
\end{equation*}
This is an offset of the Motzkin numbers.
\end{frame}
