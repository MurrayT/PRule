\chapter{Code}\label{cha:code}
    The exploratory work for this project was done mostly in the python
programming language. The code can be obtained from \url{https://github.com/MurrayT/PRule-MSc-work/tree/master/src}
and is split into the following files.

\subsubsection*{\texttt{classes.py}}
Contains the 220 equivalence classes of mesh patterns of length 2 shown in
% \ChapterRef{cha:equivs}.
\subsubsection*{\texttt{helper\_funcs.py}}
Contains functions to convert between different classes used in experimentation
as well as a class that functions as an accumulator.
\subsubsection*{\texttt{p\_rule.py}}
Contains functions that compute equivalences by the first and second dominating
rules, and functions that return coincidences not explained by the rules.
\subsubsection*{\texttt{wilf.py}}
Contains a class that functions as a counter for Wilf-equivalences based on
a dictionary, functions to calculate trivial Wilf classes and functions that
calculate Wilf-equivalences experimentally
\subsubsection*{\texttt{main.py}}
The main file to import that automatically runs all of the experimental setup
and calculations for coincidences and Wilf-equivalences.
\subsubsection*{\texttt{scratchpad.py}}
A collection of anonymous functions and utilities that do not fit into other
files.


The code relies on the the permuta python package available at \url{https://github.com/PermutaTriangle/Permuta}
% \begin{minted}[linenos, breaklines]{python}
% def boring(args=None):
%     pass
% \end{minted}
% \inputminted[linenos, breaklines, frame=lines, stripall]{python}{src/main.py}
% \lstinputlisting[language=Python]{src/main.py}
% You can put code in your document using the listings package, which is
% loaded by default in \path{custom.tex}.  Be aware that the listings
% package does not put code in your document if you are in draft mode
% unless you set the \texttt{forcegraphics} option.
%
% There is an example java (Listing~\ref{src:Data_Bus.java}) and XML
% file (Listing~\ref{src:AndroidManifest.xml}).  Thanks to the
% \texttt{url} package, you can typeset OSX and unix paths like this:
% \path{/afs/rnd.ru.is/project/thesis-template}.  Windows paths:
% \path{C:\windows\temp\ }.  You can also typeset them using the menukey
% package, but it tends to delete the last separator and has other
% complications.\footnote{The menukey package has issues with biblatex,
%   read \path{custom.tex} for more information.}
%
% If you are trying to include multiple different languages, you should
% go read the documentation and set these up in \path{custom.tex}.  You
% will save yourself a lot of effort, especially if you have to fix
% anything.
%
% %I have put the source code in the \directory{src/} folder.
% \lstinputlisting[language=Java, firstline=1,
% lastline=40, caption={Data\_Bus.java: Setting up the class.},
% label={src:Data_Bus.java}]{src/Data_Bus.java}
%
% \lstinputlisting[language={[android]XML}, firstline=1, lastline=20,
% caption={AndroidManifest.xml: Configuration for the Android UI.},
% label={src:AndroidManifest.xml}]{src/AndroidManifest.xml}

%%% Local Variables:
%%% mode: latex
%%% TeX-master: "msc-tannock-2016"
%%% End:
