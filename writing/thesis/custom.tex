%% $HeadURL$
%% $ID$
%% Various LaTeX toys for PhD Thesis
%% Collected and edited by Joe Foley<foley@MIT.EDU>

\RequirePackage[layout=inline]{fixme}
% FIXME system for making notes to each other for what needs to
% be completed in a group document.
% Alternative: ednote

% internal layouts: inline, margin, footnote, index, marginclue
% % Warning! The pdf external layouts do not play well with hyperref
% % and must be loaded by doing \fxuselayouts{}
% external layouts: marginnote, pdfnote, pdfmargin, pdfsignote, pdfsigmargin
%     pdfcnote, pdfcmargin, pdfcsignote, pdfcsigmargin
%  pdfc stands for pdf color
%\fxuselayouts{pdfcmargin}

%%%%%%%%%%%% DEFAULT PACKAGES %%%%%%%%%%%%%%%%
%% Nice package for making sub figures inside of a figure.  Keeps all
%% the reference labels correct and makes "sub captions"
% % WARNING: this is completely different than the subfigure package
\usepackage{subfig}

%% The xspace package will try to figure out if you meant to have a
%% space after a new command if you don't give it any arguments (in
%% braces: {} ).
\RequirePackage{xspace}

%% These packages are nice for drawing diagrams directly in LaTeX
\usepackage{pgf,tikz}
\usetikzlibrary{decorations.markings}

\tikzset{->-/.style={decoration={
  markings,
  mark=at position #1 with {\arrow{>}}},postaction={decorate}}}

%% This package makes typesetting chemical equations nice and simple
%% e.g. \ce{S04^2-}
%%
%% WARNING: This package sometimes causes problems due to LaTeX3 getting out of sync.
%% Make sure you run the MikTeX Updater, update FNDP, and synchronize in the package manager
%% at least at first.
%%
% \usepackage{mhchem}

% \makeatletter
% \newcounter{reaction}
% %%% >> for report and book >>
% \renewcommand\thereaction{C\,\thechapter.\arabic{reaction}}
% \@addtoreset{reaction}{chapter}
% %%% << for report and book <<
% \newcommand\reactiontag%
% {\refstepcounter{reaction}\tag{\thereaction}}
% \newcommand\reaction@[2][]%
% {\begin{equation}\ce{#2}%
% \ifx\@empty#1\@empty\else\label{#1}\fi%
% \reactiontag\end{equation}}
% \newcommand\reaction@nonumber[1]%
% {\begin{equation*}\ce{#1}\end{equation*}}
% \newcommand\reaction%
% {\@ifstar{\reaction@nonumber}{\reaction@}}
% \makeatother

%% This code makes a new reaction environment that has its own numbering for chemical formula
%% so they can be referenced in the text.  You will need the mhchem package to use it.
%% See page 12 of http://mirrors.ctan.org/macros/latex/contrib/mhchem/mhchem.pdf

\setlength{\unitlength}{1cm}
%% this is the base unit for the picture type environments
%% such as tikz

%% Force URLs to break
\setcounter{biburllcpenalty}{7000}
\setcounter{biburlucpenalty}{8000}

\RequirePackage{examplep} % another verbatim environment

% \finalifforcegraphics{listings}
% \RequirePackage{listings} % force final or included files are skipped, invisibly!
% % This allows you to include code easily
% % \begin{lstlisting}[language=bash]
% %  $ wget http://blah
% % \end{lstlisting}
% % or for simpler snippets
% %  \lstinline^cp -rp foo bar^
% %  You can replace ^ with any character not in your code
% \lstdefinestyle{default}{
%   %basicstyle=\footnotesize\ttfamily,%
%   numbers=left,%
%   numberstyle=\tiny,%
%   numberfirstline=true,%
%   stepnumber=2,%
%   numbersep=5pt,%
%   columns=fullflexible,%
%   tabsize=4,%
%   frame=lines,%
%   breaklines=true,% break long lines
%   prebreak=\raisebox{0ex}[0ex][0ex]{\ensuremath{\color{red}\hookleftarrow}}, % red arrow
%   postbreak=\raisebox{0ex}[0ex][0ex]{\ensuremath{\color{red}\hookrightarrow}}, % red arrow
%   % from http://tex.stackexchange.com/questions/116534/lstlisting-line-wrapping
% }
% \lstset{%
%   language=,%default similar to verbatim
%   style=default,
% }
% \finalifforcegraphics{minted}
% \RequirePackage{minted}

\RequirePackage{dashrule}
%% Allows us to make rules that are not just solid lines
% \hdashrule[raise][leader]{width}{height}{dash}

\RequirePackage{paralist}
%% Paragraph based list types that are generally more compact than the
%% LaTeX standard and easy to adjust
%% \setdefaultleftmargin{2em}{}{}{}{}{}  % change margins for the levels
%% \begin{asparaenum}  % enumeration
%% \begin{inparaenum} % enumerate inside of paragraph text
%% \begin{compactenum} % compressed enum
%%    replace enum with item for itemized environments
%%    replace enum with desc for descriptive environments
%%    replace enum with blank if you want formatted as if no list


%%% Typesetting algorithms and pseudocode
%% WARNING!  You cannot have both pseudocode and algorithmic packages
%% They conflict!!!!
%% http://www.macfreek.nl/memory/LaTeX_package_conflicts

%\RequirePackage{pseudocode}
%% Package for typesetting pseudocode
%\RequirePackage{algorithm}
%\RequirePackage{algorithmic}

%\RequirePackage[overload]{textcase}
%% Case changing that does not affect math
%% \MakeTextUppercase{}
%% \MakeTextLowercase{}

%% PUT YOUR PACKAGES HERE
%% If you put theme here, then it is easy to merge with any updates
%% from the main template
\usepackage{permpatts}
\usepackage{amssymb}
\usepackage{enumerate}
\usepackage{longtable}
\usepackage{tabu}
\usepackage{amsthm}
\theoremstyle{definition}
\newtheorem{theorem}{Theorem}[section]
\newtheorem{corollary}[theorem]{Corollary}
\newtheorem{proposition}[theorem]{Proposition}
\newtheorem{definition}[theorem]{Definition}
\newtheorem{lemma}[theorem]{Lemma}
\numberwithin{equation}{section}
\numberwithin{figure}{section}
\newtheorem{example}[theorem]{Example}
\newtheorem{note}[theorem]{Note}

\usepackage{mathtools}

\usepackage{xpatch}
\xpatchbibmacro{textcite}{\addspace}{\addnbspace}{}{}

%% END OF PERSONAL PACKAGES

\PassOptionsToPackage{warn}{textcomp}%
%% Turn substitutions from errors into warnings on
%% the textcomp package.
%% This is needed because some fonts do not have a \textohm for siunitx
\ifthenelse{\boolean{miktex}}{
  \RequirePackage{times}% PostScript Times font defaults
  \RequirePackage{textcomp}% special glyphs
}{
  \RequirePackage[osf]{newtxtext}
  % Replacement for mathptmx, package name is texlive-newtx
  % on RedHat, you will also need texlive-boondox
  % pg.6 http://ctan.uib.no/fonts/newtx/doc/newtxdoc.pdf
  \RequirePackage{textcomp}% special glyphs
  %% Of note, the amsmath package is loaded by newtexmath
  \RequirePackage[cmintegrals]{newtxmath}

  % \useosf % no longer required if osf specified
}

% Where do your graphics live?  If they are in this list, you don't
% need to put the path when you use \includegraphics
% Uncomment this if you need to modify the search paths.
%\graphicspath{{graphics/}{Graphics/}{./}}

%% Code listings and file paths
% Here are listings macros I use a lot
% \newcommand{\shcmd}{\lstinline}
% % for file paths use \path from the url package
% % \path{\\AFS\.rnd.ru.is\course\T-411-MECH}
%
% %% The pre-defined languages we want to use.
% \lstloadlanguages{Java, XML, Python}
%
% %% We can also define a new language (so we can change some formatting)
% %% Be careful you do not make a recursive style nor language!!
% %% You can just use the XML language, or in this case create a "dialect"
% \lstdefinelanguage[android]{XML}%
% {  %
%   sensitive=false,% case-insensitive
%   classoffset=0,  % first class
%   morekeywords={manifest},
%   classoffset=1,  % second class
%   morekeywords={uses, sdk, application, activity},
%   keywordstyle=\color{blue}, % set a color
%   classoffset=0, % reset back to 0
% }
%
% %% We use listing styles to adjust the appearance
% %% Be careful you do not make a recursive style nor language!!
% %% This makes use of the listing package to show program output
% \lstdefinestyle{progoutput}{
%   language=sh,
%   frame=single,
%   breaklines=true,
%   prebreak=\textbackslash,
%   captionpos=b,
%   basicstyle=\small\ttfamily,
%   showstringspaces=false
% }

% English/latin contractions (need italics )
\newcommand{\etal}{\textit{et al.}\xspace}
\newcommand{\aka}{\textit{a.k.a.}\xspace}
\newcommand{\adhoc}{\textit{ad hoc}\xspace}

% The abbreviation e.g. (from the Latin exempli gratia, 'for sake of an
% example') indicates that one or more examples follow of what has been
% mentioned in general terms: It could be cheaper by public transport,
% e.g. by train or coach. The abbreviation i.e. (from Latin id est,
% 'that is') indicates that an explanation follows of what has just been
% mentioned: Gratuities are discretionary, i.e. you don't have to leave
% a tip if you don't want to.
% Copyright from the Hutchinson Encyclopaedia.
% Helicon Publishing LTD 2006.
% All rights reserved.
% http://www.tiscali.co.uk/reference/dictionaries/english/data/d0081995.html
\newcommand{\ie}{\textit{i.e.}\xspace} % one specific example
\newcommand{\eg}{\textit{e.g.}\xspace} % general example

%% nicebox
% put a box around some text or a table by making
% a minipage
\newcommand{\nicebox}[1]{\fbox{\begin{minipage}[t]{\linewidth}#1\end{minipage}}}

%%% enumnoreset
%% a new list environment that does not reset unless you tell it to
\newcounter{enumnoreset}
\newcommand{\enumnoresetreset}{\setcounter{enumnoreset}{1}}
\enumnoresetreset{}

\newenvironment{enumnoreset}[1]
{ \begin{list}
    {#1\arabic{enumnoreset}.\hfil \addtocounter{enumnoreset}{1}}
    { \setlength{\listparindent}{0pt}\setlength{\itemindent}{0pt}
      \settowidth{\labelwidth}{#1 10.}
      \setlength{\leftmargin}{\labelwidth+\labelsep}
    }
  }
  { \end{list}}

%%% frlist
%% Mechanical Engineering Functional Requirements list
\newenvironment{frlist}
{\begin{enumnoreset}{FR}}
{\end{enumnoreset}}

%%% example
%% Environment for separating examples
% \newenvironment{example}
% {\begin{quote}}
% {\end{quote}}

% Abusing the \url{} function to allow SmallCaps
%\newcommand{\scurl}[1]{\def\UrlFont{\sc}\url{#1}\def\UrlFont{\tt}}
\newcommand{\scurl}[1]{{\sc \urlstyle{same}\url{#1}}}

% Force the skipping of a line (TeX code)
\newcommand{\skipline}{\vskip{\baselineskip}}
\newcommand{\skiplines}[1]{\vskip{#1*\baselineskip}}

% this is a placeholder until I figure out how to make the quotes look nice.
\newcommand{\chapterquote}[2]{%
  \begin{quote} \it``#1''#2 \end{quote}}
\newcommand{\chapterquotation}[2]{%
  \begin{quotation} \it``#1''#2 \end{quotation}}


%%% Macros to Ignore Typesetting errors when needed
\newcommand{\ignorebadness}{
  \vfuzz=10in  % ignore overfull \vbox
  \vbadness=10000 % ignore underfull \vbox
  \hfuzz=10in  % ignore overfull \hbox
  \hbadness=10000 % ignore underfull \hbox
}

\newcommand{\checkbadness}{
  \vfuzz = 0.25in
  \vbadness = 1000
  \hfuzz = 0.25in
  \hbadness = 1000
}

%%%% Various useful things from hepthesis
%%%% Andy Buckley<andy@insecnation.org>

%%% Typesets a chapter quote according to that style.

%\DeclareRobustCommand{\chapterquote}[2]{%
%  \noindent\emph{``#1''}%
%  \newline%
%  \indent --- #2%
%  \vspace{1cm}%
%}

%% In-document references
\DeclareRobustCommand{\Chapter}{Chapter\xspace}
\DeclareRobustCommand{\Section}{Section\xspace}
\DeclareRobustCommand{\Appendix}{Appendix\xspace}
\DeclareRobustCommand{\Figure}{Figure\xspace}
\DeclareRobustCommand{\Table}{Table\xspace}
\DeclareRobustCommand{\Equation}{equation\xspace}
\DeclareRobustCommand{\Reference}{reference\xspace}
\DeclareRobustCommand{\Lemma}{Lemma\xspace}
\DeclareRobustCommand{\Theorem}{Theorem\xspace}
\DeclareRobustCommand{\Proposition}{Proposition\xspace}
\DeclareRobustCommand{\Note}{Note\xspace}
%% Standard way to refer to a page number
\DeclareRobustCommand{\Page}{page\xspace}

%% Reference terms with built-in reference
\DeclareRobustCommand{\ChapterRef}[1]{\Chapter~\ref{#1}}
\DeclareRobustCommand{\SectionRef}[1]{\Section~\ref{#1}}
\DeclareRobustCommand{\AppendixRef}[1]{\Appendix~\ref{#1}}
\DeclareRobustCommand{\FigureRef}[1]{\Figure~\ref{#1}}
\DeclareRobustCommand{\TableRef}[1]{\Table~\ref{#1}}
\DeclareRobustCommand{\EquationRef}[1]{\Equation~\eqref{#1}}
\DeclareRobustCommand{\ReferenceRef}[1]{\Reference~\cite{#1}}
\DeclareRobustCommand{\PageRef}[1]{\Page~\pageref{#1}}
\DeclareRobustCommand{\LemmaRef}[1]{\Lemma~\ref{#1}}
\DeclareRobustCommand{\TheoremRef}[1]{\Theorem~\ref{#1}}
\DeclareRobustCommand{\PropositionRef}[1]{\Proposition~\ref{#1}}
\DeclareRobustCommand{\NoteRef}[1]{\Note~\ref{#1}}

%% Figure widths
\newlength{\smallfigwidth}
\newlength{\mediumfigwidth}
\newlength{\largefigwidth}
\newlength{\hugefigwidth}
\setlength{\smallfigwidth}{0.45\textwidth}
\setlength{\mediumfigwidth}{0.6\textwidth}
\setlength{\largefigwidth}{0.75\textwidth}
\setlength{\hugefigwidth}{0.9\textwidth}
%% Figure width aliases
\newlength{\littlefigwidth}
\newlength{\bigfigwidth}
\setlength{\littlefigwidth}{\smallfigwidth}
\setlength{\bigfigwidth}{\largefigwidth}

% When indexing, this sequence is very common for acronyms
\newcommand{\indexacro}[2]{\index{#2|see{#1}}\index{#1}}

%Auto incrementing variable letters
\newcounter{VarNumber}
\setcounter{VarNumber}{69}
\chardef\lastvar=68
\chardef\var=\value{VarNumber}
\newcommand{\scriptvar}{\mathcal{\var}}
\newcommand{\nextvar}[1][\lastvar]{%
\stepcounter{VarNumber}
\chardef#1=\var
\chardef\lastvar=\var
\chardef\var=\value{VarNumber}
}
%new column for tabularx
\newcolumntype{Y}{>{\centering\arraybackslash}X}
%saveable enumerate
\makeatletter
\def\saveenum{\xdef\@savedenum{\the\c@enumi\relax}}
\def\resetenum{\global\c@enumi\@savedenum}
\makeatother

%%% Local Variables:
%%% mode: latex
%%% TeX-master: "msc-tannock-2016"
%%% End:
