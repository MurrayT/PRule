Umröðun er endurröðun á \(n\) hlutum. Mengi umraðana sem forðast klassísk
umraðanamynstur geta lýst mörgum áhugaverðum eiginleikum, s.s. staflaraðanlegum
umröðunum, og tengjast mörgum fléttufræðilegum hugtökum. Möskvamynstur eru
útvíkkun á klassískum mynsturum sem leyfa auka skorður á tilvik mynstursins. Tvö
möskvamynstur eru samtilfallandi ef sömu umraðanir forðast hvert um sig. Við
finnum nægjanleg skilyrði fyrir því að tvö möskvamynstur séu samtilfallandi,
þegar umraðanir forðast einnig lengra klassískt mynstur. Þessi skilyrði, ásamt
tveimur sértilfellum, eru notuð til að gera greiningu á hvaða pör af
möskvamynstri af lengd \(2\) og klassísku mynstri af lengd \(3\) eru
samtilfallandi.

Tvö mynstur eru Wilf-jafngild ef jafn margar umraðanir af hverri lengd forðast
hvert mynstur um sig. Við Wilf-flokkum öll pör sem innihalda möskvamynstur af
lengd \(2\) og klassíska mynstrið \(231\). Að lokum finnum við ófáfengileg
Wilf-jafngildi milli pars \(231,m_1\) og \(321,m_2\), ásamt því fjalla um vinnu
sem má byggja á þessari ritgerð.
