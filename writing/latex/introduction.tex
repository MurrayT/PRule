\chapter{Introduction\label{cha:introduction}}
\section{What is a Permutation?}
In \emph{The Art of Computer Programming}\cite[p.~45]{Knuth:1997:ACP:260999} Donald Knuth defines
A \emph{permutation of \(n\) objects} is an arrangement of \(n\) distinct
objects in a row. When considering permutations we can consider them as
occuring on the set \(\nrange{n} = \setrange{1}{n}\), therefore a permutation
is a \emph{bijection} \(\pi:\nrange{n} \mapsto \nrange{n}\). We can write a
permutation \(\pi\) in two line notation
\begin{equation*}
\pi = \twoline{1,2,\dotso,n}{\pi(1),\pi(2),\dotso,\pi(n)}
\end{equation*}
However, the most frequent notation used in computer science is
\emph{one-line notation}, in this form we drop the top line of the two line
notation so are left with the following.
\begin{equation*}
\pi = \perm{\pi(1),\pi(2),\dotso,\pi(n)}
\end{equation*}

\begin{example}
There are 6 permutations on \(\nrange{3}\).
\begin{equation*}
\perm{1,2,3},\perm{1,3,2},\perm{2,1,3},\perm{2,3,1}, \perm{3,1,2}, \perm{3,2,1}
\end{equation*}
\end{example}

We can display a permutation on a \emph{figure} in order to give a graphical
representation of the permutation. In such a figure we let the \(x\)-axis
denote the index in the permutation, and the \(y\)-axis denotes the values of
\(\pi(x)\). The figure of the permutation \(\pi = \perm{2,3,1}\) is shown below
\begin{equation*}
    \pattern{}{2,3,1}{}
\end{equation*}
It is convenient to call the elements of the permutation \emph{points} when
referring to these figures.

The class of all permutations of length \(n\) is \(\mathfrak{S}_n\) and
the class has size \(n!\). The class of all permutations is
\(\mathfrak{S}=\bigcup_{i=0}^{\infty}\mathfrak{S}_i\).

\section{Classical Permutation Patterns}
Classical Permutation Patterns began to be studied as a result of Knuth's
statements about stack-sorting in \emph{The Art of Computer Programming}
\cite[p.~243, Ex.~5,6]{Knuth:1997:ACP:260999}.

\begin{definition}{Order isomorphism.}
    Two sequences \(\range{\alpha_1,\alpha_2}{\alpha_n}\) and
    \(\range{\beta_1,\beta_2}{\beta_n}\) are said to be \emph{order isomorphic}
    if they share the same relative order,i.e.,\(\alpha_r<\alpha_s\) if and
    only if \(\beta_r<\beta_s\).
\end{definition}

A permutation \(\pi\) us said to \emph{contain} the permutation \(\sigma\) of
length \(k\) as a pattern (denoted \(\sigma \preceq \pi\)) if there is some
an increasing subsequence \(\range{i_1,i_2}{i_n}\) such that the sequence
\(\range{\pi(i_1),\pi(i_2)}{\pi(i_k)}\) is order isomorphic to
\(\range{\sigma(1),\sigma(2)}{\sigma(k)}\). If \(\pi\) does not contain
\(\sigma\), we say that \(\pi\) \emph{avoids} \(\sigma\).

For example the permutation \(\pi = \perm{2,4,1,5,3}\) contains the pattern
\(\sigma = \perm{2,3,1}\), since the second, fourth and fifth entries
(\(4,5,\) and \(3\)) share the same relative order as the entries of \(\sigma\).
This can be seen graphically below, the points order isomorphic to \(\sigma\)
are highlighted.
\begin{equation*}
    \pattern{2,4,5}{2,4,1,5,3}{}
\end{equation*}

We denote the set of permutations of length \(n\) avoiding a pattern \(\sigma\)
as \(\Av_n(\sigma)\) and \(\av{\sigma} = \bigcup_{i=0}^{\infty}\Av_i(\sigma)\).

Knuth's statements were exercises in showing that the permutations avoiding the
pattern \(\perm{2,3,1}\) completely categorise permutations that are sortable to
the identity permutation using only a single stack, and that permutations avoiding the
pattern \(\perm{3,2,1}\) completely categorise permutations that are sortable to
the identity permutation using only a single queue with bypass.

\section{Mesh Patterns}
Mesh Patterns were introduced by \textcite{journals/combinatorics/BrandenC11} to capture explicit expansions
for certain permutation statistics. They are a natural extension of Classical
permutation patterns. A \emph{mesh-pattern} is a pair
\begin{equation*}
    p = (\tau,R)\text{ with } \tau \in \mathfrak{S}_k \text{ and } R \subseteq
    [0,k]\times [0,k].
\end{equation*}
By this definition the empty permutation \(\varepsilon\) as a mesh pattern consisting solely of the
box \((0,0)\).

The figure for a mesh pattern looks similar to that for a classical pattern with the
addition that we shade the unit square with bottom corner \((i,j)\) for each \((i,j) \in R\):
\begin{equation*}
    \pattern{}{ 2, 1, 3 }{ 0/1, 0/2, 1/0, 1/1, 2/1, 2/2 }
\end{equation*}

We define containment, and avoidance, of the pattern \(p\) in the permutation
\(\pi\) on mesh patterns analogously to classical containment, and avoidance,
of \(\tau\) in \(\pi\) with the additional restrictions on the relative
position of the occurence of \(\tau\) in \(\pi\). These restrictions say that
the shaded regions of the figure above contain no points from \(\pi\).

\begin{example}
    The pattern \(p=\mperm{2,1,3}{\{(0,1),(0,2),(1,0),(1,1),(2,1),(2,2)\}}=
    \textpattern{}{ 2, 1, 3 }{ 0/1, 0/2, 1/0, 1/1, 2/1, 2/2 }\) is contained in
    \(\pi = \perm{3,4,2,1,5}\) but is not contained in \(\sigma = \perm{4,2,3,1,5}\).
\end{example}
\begin{proof}
    Let us consider the figure for the permutation \(\pi\) we only need to find one occurence.
    \begin{equation*}
        \pattern{1,3,5}{3,4,2,1,5}{0/2,0/3,0/4,
                                   1/0,1/1,1/2,
                                   2/0,2/1,2/2,
                                   3/2,3/3,3/4,
                                   4/2,4/3,4/4}
    \end{equation*}
    We have found an occurence of the pattern \(p\) in \(\pi\) and therefore
    \(\pi\) contains \(p\).

    Now we consider the figure for the permutation \(\sigma\). This permutation
    avoids \(p\) since for every occurence of the classical pattern \(\perm{2,1,3}\)
    there is at least one point in one of the shaded boxes. Consider the subsequence
    \(315\) in \(\sigma\), this is an occurence of \(\perm{2,1,3}\) but not the mesh
    pattern since the points with values \(4\) and \(2\) are in the shaded areas.
    This is shown in the figure below.

    \begin{equation*}
        \pattern{3,4,5}{4,2,3,1,5}{0/1,0/2,0/3,0/4,
                                   1/1,1/2,1/3,1/4,
                                   2/1,2/2,2/3,2/4,
                                   3/0,3/1,3/2,
                                   4/1,4/2,4/3,4/4}
    \end{equation*}
    This is true for all occurences of \(\perm{2,1,3}\) in \(\sigma\) and
    therefore \(\sigma\) avoids \(p\).
\end{proof}

We denote the avoidance sets for mesh patterns in the same way as for
classical patterns.

\begin{note}
    \label{not:classmesh}
    Classical patterns are just a subclass of mesh patterns where the mesh
    set \(R\) is empty. The classical pattern \(\pi\) can be represented
    by a mesh pattern as \((\pi,\emptyset)\).
\end{note}

In the past people have studied different classes
of permutations that can be described by mesh patterns. \textcite{babstein2000}
considered \emph{vincular} patterns (also known as \emph{generalised} or \emph{dashed} patterns),
those where two adjacent entries in the pattern must be adjacent in the permutation,
\ie entire columns are shaded in the mesh.
\textcite{MR2652101} look at classes of pattern where both columns and rows
can be shaded, these are called \emph{bivincular} patterns.
\emph{Bruhat-restricted} patterns were studied by \textcite{MR2264071} in order
to establish necessary conditions for a Schubert variety to be Gorenstein.
Mesh patterns also encompass a subset of \emph{barred} patterns introduced by
\textcite{MR2716312}.

Avoiding pairs of patterns of the same length with certain properties has
also been studied in the past, \textcite{MR2178749} considered avoiding a
pair of vincular patterns of length 3. \textcite{2015arXiv151203226B} study avoiding
a vincular and a covincular pattern simultaneously in order to achieve some
interesting counting results. However, very little work has been done on avoiding
a mesh pattern and a classical pattern simultaneously, in this work we aim to
establish some ground in this field by computing coincidences and Wilf-classes
and calculating some of the enumerations of avoiders of a mesh pattern of length
2 and a classical pattern of length 3.

%%% Local Variables:
%%% mode: latex
%%% TeX-master: "msc-tannock-2016"
%%% End:
