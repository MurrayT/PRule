\section{Dominating pattern rules}
\begin{table}[htb]
\begin{center}
\begin{tabularx}{\textwidth}{c|Y|Y|Y|Y|}
\cline{2-5}
& \multicolumn{4}{|c|}{Dominating Pattern}\\
\cline{2-5}
& \multicolumn{2}{|c|}{\(\perm{2,3,1}\)}&\multicolumn{2}{|c|}{\(\perm{3,2,1}\)}\\
\cline{2-5}
& \(\perm{1,2}\)&\(\perm{2,1}\)&\(\perm{1,2}\)&\(\perm{2,1}\)\\
\hline
\multicolumn{1}{|r|}{No Dominating rule}&\(220\)&\(220\)&\(220\)&\(220\)\\
\hline
\multicolumn{1}{|r|}{First Dominating rule}&\(85\)&\(43\)&\(220\)&\(29\)\\
\hline
\multicolumn{1}{|r|}{Second Dominating rule}&\(59\)&\(39\)&\(220\)&\(29\)\\
\hline
\multicolumn{1}{|r|}{Third Dominating rule}&\(56\)&\(39\)&\(220\)&\(29\)\\
\hline
\multicolumn{1}{|r|}{Experimental class size}&\(56\)&\(39\)&\(213\)&\(29\)\\
\hline
\end{tabularx}
\end{center}
    \caption{Class number reduction by application of Dominating Rules}
\end{table}
