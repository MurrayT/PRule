\chapter{Conclusions and Future work}
\nextvar
% From \ChapterRef{chap:coincs} it can be seen that automatically classifying
% coincidences of mesh patterns is a difficult task, establishing rules for longer
% dominating patterns requires many more cases to be taken.
If we consider a similar system for dominating patterns of length \(4\) and
mesh patterns of length \(2\), it can be seen that the number of cases required
to establish rules increases to a number that is infeasible to compute manually.
For an extension of the First Dominating rule alone, we would have to consider
placement of points in any pair of unshaded regions. The fact that the rules
established do not completely cover the coincidences with a dominating pattern
of length \(3\) (see \TableRef{tab:domclasses}) shows that this is a difficult
task.

It is interesting to consider the application of the Third Dominating rule,
as well as the simple extension of allowing a sequence of \texttt{add\_point}
operations, to meshpatterns without any dominating pattern in order to try to
capture some of the coincidences described in
\textcite{DBLP:journals/combinatorics/HilmarssonJSVU15}
 and \textcite{DBLP:journals/corr/ClaessonTU14}.
 \begin{example}
   We can establish the coincidence between the patterns
   \begin{equation*}
     m_1 = \pattern{}{1,2}{0/1,0/2,1/1,1/2,2/0}, \text{ and } m_2 = \pattern{}{1,2}{0/2,1/0,1/1,2/0,2/1}
   \end{equation*}
   That is not explained by the methods presented by \textcite{DBLP:journals/corr/ClaessonTU14}.

   Consider a permutation containing \(m_1\),
   \begin{equation*}
     \begin{tikzpicture}[scale=0.5]
         \modpattern[5]{}{1,2}{0/1,0/2,1/1,1/2,2/0}
         \draw (1.5,0.5) node {\(Y\)};
         \draw (2.5,1.5) node {\(X\)};
     \end{tikzpicture}
   \end{equation*}
   If the regions corresponding to both \(X\) and \(Y\) are empty then we have
   an occurrence of \(m_2\).
   Consider if the region corresponding to \(X\) is non-empty, we can then choose
   the lowest valued point in this region
   \begin{equation*}
     \begin{tikzpicture}[scale=0.5]
         \modpattern[5]{1,3}{1,3,2}{0/1,0/2,0/3,1/1,1/2,1/3,2/0,2/1,3/0,3/1}
         \draw (1.5,0.5) node {\(Y\)};
     \end{tikzpicture}
   \end{equation*}
   If the region corresponding to \(Y\) is empty then we have an occurrence of
   \(m_2\) with the indicated points.
   Now if the region corresponding to \(Y\) is non-empty, we can choose the
   rightmost point in this region.
   \begin{equation*}
     \begin{tikzpicture}[scale=0.5]
         \modpattern[5]{2,4}{2,1,4,3}{0/2,0/3,0/4,1/2,1/3,1/4,2/0,2/1,2/2,2/3,2/4,3/0,3/1,3/2,4/0,4/1,4/2}
     \end{tikzpicture}
   \end{equation*}
   And now the two indicated points form an occurrence of \(m_2\).
   We have therefore shown that any occurrence of \(m_1\) is an occurrence of
   \(m_2\) and we can easily show the converse by the same reasoning, so \(m_1\)
   and \(m_2\) are coincident.
   This is captured by an extension of the Third Dominating rule where we allow
   multiple steps of adding points before we check for subpattern containment.
 \end{example}
It is not possible to apply the first and second Dominating Rules to the pattern
itself, since when applying the rules we consider containers of the pattern inside
the avoiders of the dominating pattern. For example if we were to attempt to apply
the first rule to the pattern \(\perm{1,2}\) then we would have to consider
containers of \(\perm{1,2}\) inside \(\av{\perm{1,2}}\), and obviously this can
never occur.

It is also possible to take sets of mesh patterns instead of a single mesh pattern
when considering dominating rules, and expressing coincidence between these sets.
Doing this may give nice results. Coincidences between sets with
multiple dominating patterns can also be considered, as this provides even more
power to the rules discussed, these methods may be useful for classifying coincidence
in large sets of pattern. It is also possible to apply these rules to sets where the
dominating pattern is a mesh pattern.

It would be interesting to consider a systematic explanation of Wilf-equivalences
amongst classes where \(\perm{3,2,1}\) is the dominating pattern, possible using the
construction presented in \cite[Sec.~12]{2015arXiv151203226B}, in order to directly
reach enumeration and hopefully establish some of the non-trivial Wilf-equivalences
between classes with different dominating patterns. For example, it is possible to show that
the sets \(\scriptvar = \av{\textpattern{}{1,2}{1/2,2/0,0/0,1/0,1/1,2/1},\perm{2,3,1}}\)\nextvar and
\(\scriptvar = \av{\textpattern{}{1,2}{1/2,2/0,0/0,1/0,1/1,2/1},\perm{3,2,1}}\), are Wilf-equivalent.
This can be seen by considering an occurrence of the mesh pattern,
\(p=\textpattern{}{1,2}{1/2,2/0,0/0,1/0,1/1,2/1}\) in a permutation
\begin{equation*}
    \raisebox{0.6ex}{
    \begin{tikzpicture}[scale=1, baseline=(current bounding box.center)]
        \draw (0.5,1.5) node {\(b_1\)};
        \draw (0.5,2.5) node {\( b_2\)};
        \draw (2.5,2.5) node {\( b_3\)};
        \modpattern[2]{2}{1,2}{0/0,1/0,1/1,1/2,2/0,2/1}
    \end{tikzpicture}
    }
\end{equation*}

If we are inside \(\av{\perm{2,3,1}}\) then any points in the region corresponding
to the box \(b_1\) must be to the right of the points in the region corresponding to
box \(b_2\), and must also form a decreasing subsequence by \LemmaRef{lem:incdec}.
Furthermore, the points in the regions corresponding to \(b_2\) and
\(b_3\) must form an avoider of \(\perm{2,3,1}\) with the indicated point.
Therefore the containers of \(p\) in \(\av{\perm{2,3,1}}\) have generating function
\(\lastvar(x) = x(C(x)-1)/(1-x)\).
Now, if we are in \(\av{\perm{3,2,1}}\) then any points in the region corresponding
to the box \(b_1\) must be to the left of the points in the region corresponding to
box \(b_2\), and must also form a increasing subsequence by \LemmaRef{lem:incdec}.
Furthermore, the points in the regions corresponding to \(b_2\) and
\(b_3\) must form an avoider of \(\perm{3,2,1}\) with the indicated point.
Hence, the containers of \(p\) in \(\av{\perm{3,2,1}}\) have generating function
\(\var(x) = x(C(x)-1)/(1-x)\) and both of these classes have the same
enumeration.
