\chapter{Conclusions and Future work}
From \ChapterRef{chap:coincs} it is can be seen that automatically classifying
coincidences of mesh patterns is a difficult task, establishing rules for longer
dominating patterns requires many more cases to be taken. It would however
be interesting to consider the self application of the  third Dominating Rule to mesh
patterns in order to try to capture some of the coincidences described in
\textcite{DBLP:journals/combinatorics/HilmarssonJSVU15} and \textcite{journals/combinatorics/BrandenC11}.
It is not possible to apply the first and second Dominating Rules to the pattern
itself, since when applying the rules we consider containers of the pattern inside
the avoiders of the dominating pattern. For example if we were to attempt to apply
the first rule to the pattern \(\perm{1,2}\) then we would have to consider
containers of \(\perm{1,2}\) inside \(\av{\perm{1,2}}\), and obviously this can
never occur.

It is also possible to take sets of mesh patterns instead of a single mesh pattern
when considering dominating rules, and expressing coincidence between these sets.
Doing this may give nice enumerative results. Coincidences between sets with
multiple dominating patterns can also be considered, as well as sets where the
dominating pattern is a mesh pattern.

It would be interesting to consider a systematic explanation of Wilf-equivalences
amongst classes where \(\perm{3,2,1}\) is the dominating pattern using the
construction presented in \cite[Sec.~12]{2015arXiv151203226B}, in order to directly
reach enumeration and hopefully establish some of the non-trivial Wilf-equivalences
between classes with different dominating patterns. It is possible to show that
the sets consisting of \textpattern{}{1,2}{1/2,2/0,0/0,1/0,1/1,2/1} and
\(\perm{2,3,1}\), or \(\perm{3,2,1}\), are Wilf-equivalent.
