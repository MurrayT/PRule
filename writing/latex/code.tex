\chapter{Code}\label{cha:code}
You can put code in your document using the listings package, which is
loaded by default in \path{custom.tex}.  Be aware that the listings
package does not put code in your document if you are in draft mode
unless you set the \texttt{forcegraphics} option.

There is an example java (Listing~\ref{src:Data_Bus.java}) and XML
file (Listing~\ref{src:AndroidManifest.xml}).  Thanks to the
\texttt{url} package, you can typeset OSX and unix paths like this:
\path{/afs/rnd.ru.is/project/thesis-template}.  Windows paths:
\path{C:\windows\temp\ }.  You can also typeset them using the menukey
package, but it tends to delete the last separator and has other
complications.\footnote{The menukey package has issues with biblatex,
  read \path{custom.tex} for more information.}

If you are trying to include multiple different languages, you should
go read the documentation and set these up in \path{custom.tex}.  You
will save yourself a lot of effort, especially if you have to fix
anything.

%I have put the source code in the \directory{src/} folder.
\lstinputlisting[language=Java, firstline=1,
lastline=40, caption={Data\_Bus.java: Setting up the class.},
label={src:Data_Bus.java}]{src/Data_Bus.java}

\lstinputlisting[language={[android]XML}, firstline=1, lastline=20,
caption={AndroidManifest.xml: Configuration for the Android UI.},
label={src:AndroidManifest.xml}]{src/AndroidManifest.xml}

%%% Local Variables: 
%%% mode: latex
%%% TeX-master: "msc-tannock-2016"
%%% End: 
