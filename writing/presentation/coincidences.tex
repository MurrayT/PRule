\section{Coincidence Classes}
\label{sec:Coincidence Classes}
\subsection{Coincidence}
\label{sub:Coincidence}
\begin{frame}{Coincidence}
  \begin{definition}
    Two mesh patterns are said to be \emph{coincident} if they avoid the
    same set of patterns at every length.
  \end{definition}
  \begin{block}{}
    Classical patterns can never be coincident.

    Aim to establish rules that classify coincidences when we
    have one mesh pattern and one classical pattern.
  \end{block}
\end{frame}

\subsection{Summary of Experimental Results}
\label{sub:Summary of Experimental Results}
\begin{frame}{Experimental Results}
  \begin{block}{}
  \begin{table}[htb]
\begin{center}
\begin{tabularx}{\textwidth}{c|Y|Y|Y|Y|}
\cline{2-5}
& \multicolumn{4}{|c|}{Dominating Pattern}\\
\cline{2-5}
& \multicolumn{2}{|c|}{\(\perm{2,3,1}\)}&\multicolumn{2}{|c|}{\(\perm{3,2,1}\)}\\
\cline{2-5}
& \(\perm{1,2}\)&\(\perm{2,1}\)&\(\perm{1,2}\)&\(\perm{2,1}\)\\
\hline
\multicolumn{1}{|r|}{No Dominating rule}&\(220\)&\(220\)&\(220\)&\(220\)\\
\hline
\multicolumn{1}{|r|}{First Dominating rule}&\(85\)&\(43\)&\(220\)&\(29\)\\
\hline
\multicolumn{1}{|r|}{Second Dominating rule}&\(59\)&\(39\)&\(220\)&\(29\)\\
\hline
\multicolumn{1}{|r|}{Third Dominating rule}&\(56\)&\(39\)&\(220\)&\(29\)\\
\hline
\multicolumn{1}{|r|}{Experimental class size}&\(56\)&\(39\)&\(213\)&\(29\)\\
\hline
\end{tabularx}
\end{center}
    \caption{Coincidence class number reduction by application of Dominating rules}
    \label{tab:domclasses}
\end{table}
\end{block}
\end{frame}

\subsection{Dominating Pattern Rules}
\label{sub:Dominating pattern Rules}

\begin{frame}{First Dominating rule}
  \begin{block}{Proposition: First Dominating rule}
      \label{prop:dom1}
      \emph{
      Given two mesh patterns \(m_1 =(\sigma, R_1)\) and \(m_2 = (\sigma, R_2)\),
      and a dominating classical pattern \(\pi = (\pi,\emptyset)\) such that
      \(\setsize{\pi} \le \setsize{\sigma} + 1\), the sets \(\av{\{\pi,m_1\}}\) and
      \(\av{\{\pi,m_2\}}\) are coincident if

      \begin{enumerate}
          \item \(R_1 \triangle R_2 = \{(a,b)\}\)
          \item \(\pi \preceq \addp{\sigma}{(a,b)}\)\label{prop:dom1:cont}
      \end{enumerate}}
  \end{block}
\end{frame}

\begin{frame}
  \begin{example}
    The following two patterns are coincident in \(\av{\perm{3,2,1}}\)
    \begin{equation*}
      \begin{tikzpicture}[scale=\picscale]
        \modpattern{}{2,1}{1/0,1/1,1/2,2/2}
    \end{tikzpicture}
        \raisebox{2ex}{\qquad}
    \begin{tikzpicture}[scale=\picscale]
        % \filldraw (2.5,0.5) circle (3pt);
        \modpattern{}{2,1}{1/0,1/1,1/2,2/2,2/0}
    \end{tikzpicture}
    \end{equation*}
  \end{example}
  \begin{corollary}
    All coincidences of classes the form \(\av{\{\perm{3,2,1},\mperm{2,1}{R}}\}\)
    are fully explained by the First Dominating rule.
  \end{corollary}
  \begin{block}{}
    There are \(29\) coincidences of mesh patterns of the form \(\av{\{\perm{3,2,1},\mperm{2,1}{R}}\}\)
  \end{block}
\end{frame}

\begin{frame}{Second Dominating rule}
  \begin{block}{}
    The patterns
    \begin{equation*}
    m_1 = \pattern{}{2,1}{0/0,0/2,1/0,2/1} \text{ and } m_2 = \pattern{}{2,1}{0/0,0/2,1/0,1/1,2/1}
    \end{equation*}
    are coincident in \(\av{\perm{2,3,1}}\).
  \end{block}
\end{frame}

\begin{frame}
  \begin{lemma}
    Given a mesh pattern \(m =(\sigma, R)\), where the box \((a,b)\) is not
in \(R\), and a dominating classical pattern \(\pi = (\pi,\emptyset)\) if
\(\pi \preceq \adda{\sigma}{(a,b)}\)\\(\(\pi \preceq \addd{\sigma}{(a,b)}\))
,then in any occurrence of \(m\) in a permutation \(\varrho\), the region
corresponding to the box \((a,b)\) can only contain an increasing
(decreasing) subsequence of \(\varrho\).
  \end{lemma}
\end{frame}

\begin{frame}
  \begin{example}
  Considering \(m_1\) again
  \begin{equation*}
    \begin{tikzpicture}[scale=\picscale]
        {\only<3-> \fill[pattern=north east lines, pattern color=black!75] (1,1) rectangle +(1,1);}
        \modpattern{}{2,1}{0/0,0/2,1/0,2/1}
        {\only<2> \filldraw (1.66,1.33) circle (3pt);}
        {\only<2> \filldraw (1.33,1.66) circle (3pt);}
    \end{tikzpicture}
  \end{equation*}
  {\only<1-3> \vphantom{This is \(m_2\).}}
  {\only<4-> {}}
  \end{example}
\end{frame}

\begin{frame}
  \begin{block}{Proposition: Second Dominating rule}
    \label{prop:dom2}
    \emph{
    Given two mesh patterns \(m_1 =(\sigma, R_1)\) and \(m_2 = (\sigma, R_2)\),
    and a dominating classical pattern \(\pi = (\pi,\emptyset)\) such that
    \(\setsize{\pi} \le \setsize{\sigma} + 2\), the sets \(\av{\{\pi,m_1\}}\) and
    \(\av{\{\pi,m_2\}}\) are coincident if

    \begin{enumerate}
        \item \(R_1 \triangle R_2 = \{(a,b)\}\)
        \item   \begin{enumerate}
                \item\label{prop:dom2:condc} \(\pi \preceq \adda{\sigma}{(a,b)}\) and
                        \begin{enumerate}
                            \item \((a+1,b) \in \sigma\) and \((a+1,b-1)\notin R\) and \\
                                \((x,b-1)\in R \implies (x,b) \in R \) (where \(x\neq a,a+1\)) and\\
                                  \((a+1,y)\in R \implies (a,y) \in R\) (where \(y\neq b-1,b\)).
                            \item \(\dots\)
                        \end{enumerate}
                    \item \(\dots\)\begin{enumerate}
                            \item \(\dots\)
                            \item \(\dots\)
                          \end{enumerate}
                \end{enumerate}
    \end{enumerate}
    }
  \end{block}
\end{frame}

\begin{frame}
  \begin{example}
    \begin{figure}[ht]
\centering
\begin{tikzpicture}[scale = 0.75]

     % Make the grid
     \draw[step=1cm,gray!50] (2/3,2/3) grid (5+1/3,5+1/3);

     % Make black cross in the middle
     \draw[-,very thick] (3,2/3) -- (3,5+1/3);
     \draw[-,very thick] (2/3,3) -- (5+1/3,3);
     \draw[-,very thick,gray] (4,2) -- (2,2) -- (2,4) -- (4,4) -- (4,2);

     \fill[black] (3,3) circle (0.1);
     \fill[red] (2,3) circle (0.1);

     % Draw the dots on the side
     \foreach \x in {1,2,4,5}{
         \draw[dotted, gray!50] (\x, 0) -- (\x, 2/3);
         \draw[dotted, gray!50] (\x, 5+1/3) -- (\x, 6);
         \draw[dotted, gray!50] (0,\x) -- (2/3,\x);
         \draw[dotted, gray!50] (5+1/3,\x) -- (6,\x);
        }
     % Draw black dots next to the cross
     \draw[dotted, black, very thick] (3, 0) -- (3, 2/3);
     \draw[dotted, black, very thick] (3, 5+1/3) -- (3, 6);
     \draw[dotted, black, very thick] (0, 3) -- (2/3, 3);
     \draw[dotted, black, very thick] (5+1/3, 3) -- (6, 3);

     \draw[->-=.5,thick] (3,3) to[bend right=15](2,4);

     \draw[->, thick] (5,3) -- (5,4);
     \draw[->, thick] (1,3) -- (1,4);
     \draw[->, thick] (3,5) -- (2,5);
     \draw[->, thick] (3,1) -- (2,1);

     \node [black,right] at (6,3) {$b$};
     \node [black,below] at (2,0) {$a$};

\end{tikzpicture}
    \caption{If the conditions of The Second Dominating rule are satisfied the box
    \((a-1,b)\) can be shaded.}
\end{figure}
  \end{example}
\end{frame}

\begin{frame}
  \begin{corollary}
    All coincidences of classes the form \(\av{\{\perm{2,3,1},\mperm{2,1}{R}}\}\)
    are fully explained by applying the First Dominating rule, then applying the
    Second Dominating rule.
  \end{corollary}
  \begin{block}{}
    There are \(39\) coincidences of mesh patterns of the form
    \(\av{\{\perm{2,3,1},\mperm{2,1}{R}}\}\)
  \end{block}
\end{frame}

\begin{frame}{Third Dominating rule}
  \begin{block}{}
    The patterns
    \begin{equation*}
    m_1 = \pattern{}{1,2}{0/2,2/0,2/1} \text{ and } m_2 = \pattern{}{1,2}{0/2,1/0,2/0,2/1}
    \end{equation*}
    are coincident in \(\av{\perm{2,3,1}}\).
    Neither of the previous two rules explain this.
  \end{block}
\end{frame}

\begin{frame}
  \begin{example}
    \begin{equation*}
    \only<1>{\vphantom{\pattern{2,3}{2,1,3}{0/3,1/3,2/0,2/1,3/0,3/1,3/2}}\pattern{}{1,2}{0/2,2/0,2/1}}
    \only<2>{\pattern{}{2,1,3}{0/3,2/0,2/1,3/0,3/1,3/2}}
    \only<3>{\pattern{}{2,1,3}{0/3,1/3,2/0,2/1,3/0,3/1,3/2}}
    \only<4>{\pattern{2,3}{2,1,3}{0/3,1/3,2/0,2/1,3/0,3/1,3/2}}
    \only<5->{\vphantom{\pattern{2,3}{2,1,3}{0/3,1/3,2/0,2/1,3/0,3/1,3/2}}\pattern{}{1,2}{0/2,1/0,2/0,2/1}}
  \end{equation*}
  \end{example}
\end{frame}

\begin{frame}{}
  \begin{block}{Proposition: Third Dominating rule}
      \emph{
      Given two mesh patterns \(m_1 =(\sigma, R_1)\) and \(m_2 = (\sigma, R_2)\),
      and a dominating classical pattern \(\pi = (\pi,\emptyset)\), the sets
      \(\av{\{\pi,m_1\}}\) and \(\av{\{\pi,m_2\}}\) are coincident if
      \begin{enumerate}
          \item \(R_1 \triangle R_2 = \{(a,b)\}\)
          \item\label{prop:dom3:condocc} \(\addp[D]{\mperm{\sigma}{R_1}}{(a,b)}\)
              where \(D\in\{N,E,S,W\}\)
              is coincident with a mesh pattern containing an occurrence of
              \(\mperm{\sigma}{R_2}\) as a subpattern.
      \end{enumerate}
      }
  \end{block}
\end{frame}

\begin{frame}
  \begin{corollary}
    All coincidences of classes the form \(\av{\{\perm{2,3,1},\mperm{1,2}{R}}\}\)
    are fully explained by applying the First Dominating rule, the
    Second Dominating rule, and then the Third Dominating rule.
  \end{corollary}
  \begin{block}{}
    There are \(56\) coincidences of mesh patterns of the form
    \(\av{\{\perm{2,3,1},\mperm{2,1}{R}}\}\)
  \end{block}
\end{frame}
